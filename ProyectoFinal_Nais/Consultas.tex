\documentclass[12pt,letterpaper]{article}
\newcommand*{\plogo}{\fbox{$\mathcal{PL}$}}
\usepackage[a4paper]{geometry}
\geometry{left=1.7cm,right=2cm,top=2.2cm,bottom=3cm}
\usepackage[utf8]{inputenc}
\usepackage[T1]{fontenc}
\usepackage{stackengine}
\usepackage{graphicx}
\usepackage{multirow}
\usepackage{ragged2e}
\usepackage{enumitem}
\usepackage{amsmath}
\usepackage{ wasysym }
\usepackage{fancyhdr}
\usepackage[dvipsnames]{xcolor}
%Comandos para las cajitas TO DO LIST
\usepackage{enumitem,amssymb}
\newlist{todolist}{itemize}{2}
\setlist[todolist]{label=$\square$}
\usepackage{pifont}
\newcommand{\cmark}{\ding{51}}%
\newcommand{\xmark}{\ding{55}}%
\newcommand{\done}{\rlap{$\square$}{\raisebox{2pt}{\large\hspace{1pt}\cmark}}%
\hspace{-2.5pt}}
\newcommand{\wontfix}{\rlap{$\square$}{\large\hspace{1pt}\xmark}}
\usepackage{tcolorbox}
\tcbuselibrary{listingsutf8}
\usepackage{array}
\usepackage[dvipsnames]{xcolor}
\usepackage{amssymb}
\usepackage[spanish]{babel}
\usepackage[fixlanguage]{babelbib}
    \bibliographystyle{babunsrt}
\usepackage{url}  
\usepackage[dvipsnames]{xcolor}
\usepackage{ifsym}
\usepackage{amssymb}
\usepackage{hyperref}

\begin{document}

\pagestyle{fancy}
\fancyhf{}
\fancyhead[C]{}
\rhead {Proyecto Final: Centro de Contactos} 
\lhead {Consultas} 
\rfoot{\thepage}
\begin{center}
   \section*{Consultas diseñadas} 
\end{center}

\begin{enumerate}
    \item \textbf{Información de los cursos que se tienen activos} \\
    Consulta:\\
    \textsc{ SELECT * FROM curso WHERE fecha\_fin < (SELECT NOW() );}\\devuelve los siguientes datos de los cursos que se encuentran activos.
\begin{itemize}
        \item $id\_curso$
        \item $id\_entrenador$
        \item $id\_sala\_entrenamiento$
        \item $id\_piso$
        \item $id\_edificio$
        \item $id\_cliente$
        \item $nombre\_curso$
        \item $turno$
        \item $fecha\_inicio$
        \item $fecha\_fin$
        \item $modalidad$
        \item $num\_horas$
    \end{itemize}
\begin{figure}[h!]
    \centering
    \includegraphics[scale=.4]{PROYECTOFINAL/Consultas/1.png}
\end{figure}

\newpage
    \item \textbf{Entrenadores que se encuentran dando un curso} \\
    Consulta:\\ \textsc{SELECT * FROM (entrenador LEFT JOIN curso\\ ON entrenador.curp=curso.id\_entrenador)\\ WHERE (id\_curso IS NOT NULL);} \\Devuelve los 35 entrenadores  que están dando curso con sus respectivos datos, lo que son\\
    \texttt{CURP, Id\_piso, Id\_edificio, nombre, apellido\_pat, apellido\_mat, fecha\_nacimiento,} dirección \textttt{(calle, num\_exterior, colonia, municipio), estado, cp }(Código Postal), \texttt{telefono, correo, fotografia, salario\_x\_hora, nss} (Número de Seguridad Social), \texttt{fecha\_inicio,\\Id\_curso, Id\_entrenador,Id\_sala\_entrenamiento, Id\_piso, Id\_edificio} (los cuales se muestran nulos debido a que son impartidos en línea), \texttt{Id\_cliente, nombre\_curso, turno, fecha\_inicio, fecha\_fin, modalidad, num\_horas}.
    \begin{figure}[h]
        \centering
        \includegraphics[scale=.35]{PROYECTOFINAL/Consultas/2.1.png}
    \end{figure}
    \begin{figure}[h]
        \centering
        \includegraphics[scale=.38]{PROYECTOFINAL/Consultas/2.2.png}
    \end{figure}
    \begin{figure}[h]
        \centering
        \includegraphics[scale=.35]{PROYECTOFINAL/Consultas/2.3.png}
    \end{figure}

\newpage
    \item \textbf{Cantidad de horas máximas de los cursos}\\     
    Consulta:\\
    \textsc{ SELECT MAX (num\_horas) FROM curso;}
    \begin{figure}[h]
        \centering
        \includegraphics[scale=.5]{PROYECTOFINAL/Consultas/3.png}
    \end{figure}

    
    \item \textbf{El historial de cursos de los clientes en específico}\\
    Se crea una tabla historial\_curso que contiene los datos que nos interesan almacenar
    \begin{figure}[h!]
      \centering
      \includegraphics[scale=.5]{PROYECTOFINAL/Consultas/4.1.png}
  \end{figure}\\
  Consulta:\\ \textsc{ SELECT * FROM (SELECT * FROM historial\_cursos\\ WHERE id\_cliente='DODD3310206E5') AS A UNION
	 (SELECT id\_curso, \\id\_cliente,nombre\_curso,turno,fecha\_inicio,fecha\_fin FROM curso \\WHERE id\_cliente='DODD3310206E5' OR\ id\_cliente='HAQD979294S1' OR\ id\_cliente='SKFF767755A2');}\\
 Que como resultado devuelve el historial de cursos los clientes con el ID:
\begin{align*}
    &DODD3310206E5,\\&HAQD979294S1,\\&SKFF767755A2
\end{align*}
De los cuales solo se encuentran 5 resultados en la Base de Datos
\begin{figure}[h]
    \centering
    \includegraphics[scale=.5]{PROYECTOFINAL/Consultas/4.2.png}
\end{figure}
\newpage
    \item  \textbf{Nombre del agente\_telefonico con evaluación menor a 8} \\
    Consulta:\\ \textsc{ SELECT nombre FROM agente\_telefonico\\ WHERE (agente\_telefonico.evaluacion < 8);}\\ Regresa exactamente los 25 agentes telefónicos donde se nos presenta su respectivo nombre con  evaluación no aprobatoria, es decir, menor a 8.

    \begin{figure}[h]
        \centering
        \includegraphics[scale=.5]{PROYECTOFINAL/Consultas/5.png}
    \end{figure}

    
    \item \textbf{Salas de operaciones que se encuentran disponibles} \\
     Consulta:\\ \textsc{ SELECT id\_piso FROM sala\_operaciones\\WHERE sala\_operaciones.estado = 0;}\\Devuelve los 19 ID's de los pisos disponibles.
    \begin{figure}[h]
        \centering
        \includegraphics[scale=.5]{PROYECTOFINAL/Consultas/6.png}
    \end{figure}

   \newpage 
    \item \textbf{Agentes telefónicos que han nacido antes del 2000} \\
     Consulta:\\ \textsc{ SELECT * FROM agente\_telefonico WHERE EXTRACT(YEAR FROM agente\_telefonico.fecha\_nacimiento) < '2000';}
    \begin{figure}[h]
        \centering
        \includegraphics[scale=.38]{PROYECTOFINAL/Consultas/7.1.png}\\
         \includegraphics[scale=.4]{PROYECTOFINAL/Consultas/7.2.png}\\          \includegraphics[scale=.36]{PROYECTOFINAL/Consultas/7.3.png}
    \end{figure}\\
    Existen 89 agentes telefónicos que nacieron antes del 2000, y en la consulta aparecen sus datos personales como el $id\_piso$, $id\_edificio$ y $id\_curso$ en el que operan.
    
    \item \textbf{Clientes con sistema operativo \textit{Linux}}\\
    Consulta:\\ \textsc{ SELECT rfc FROM cliente WHERE cliente.so = 'Linux';}\\ Nos devuelve en total 44 clientes con el  sistema operativo \textit{Linux}, en la tabla solo se mostrará su respectivo rfc.
    \begin{figure}[h]
        \centering
        \includegraphics[scale=.36]{PROYECTOFINAL/Consultas/8.png}
    \end{figure}
%% 9<
    \item \textbf{Datos de los entrenadores que viven en CDMX}\\
  Consulta:\\  \textsc{ SELECT nombre,apellido\_pat,apellido\_mat FROM entrenador WHERE entrenador.estado = 'CDMX';} \\ Devuelve 26 valores, nombre, apellido\_pat y apellido\_mat de los entrenadores que viven en la CDMX.  \begin{figure}[h!]
        \centering
        \includegraphics[scale=.5]{PROYECTOFINAL/Consultas/9.png}
    \end{figure}

    \item \textbf{Horarios en el día \textit{Viernes}} \\
    Consulta:\\ \textsc{ SELECT * FROM horario WHERE horario.dia = 'Viernes';} \\Devuelve el ID de los horarios, ID del curso, el día que corresponde, la hora de inicio y su hora final solamente de los días Viernes. Existen 11 valores con esta descripción, pero solo adjuntamos los 5 primeros.
     \begin{figure}[h]
         \centering
         \includegraphics[scale=.5]{PROYECTOFINAL/Consultas/10.png}
     \end{figure}


    \newpage
    \item \textbf{Entrenadores y cursos en el turno matutino} \\
   Consulta:\\  \textsc{ SELECT id\_entrenador,nombre\_curso FROM curso\\ WHERE curso.turno = 'Matutino';}\\
    Existen 15 entrenadores y sus respectivos cursos con el horario matutino y el nombre del mismo curso. Adjuntamos los primeros 7 que aparecen. 
    \begin{figure}[h]
        \centering
        \includegraphics[scale=.5]{PROYECTOFINAL/Consultas/11.png}
    \end{figure}

    %%%% 11
    
    \item \textbf{Estaciones de operaciones en el edificio con ID=10} \\
    Consulta:\\ \textsc{ SELECT * FROM (estacion\_operaciones\\ NATURAL JOIN aditamento\_operaciones) WHERE id\_edificio = 10;}\\
     Existen 12 diferentes estaciones de operaciones con el mismo ID de edificio (10), aparece su respectivo sistema operativo, el nombre, aditamiento y lo que se muestra a continuación.
     \begin{figure}[h]
         \centering
         \includegraphics[scale=.4]{PROYECTOFINAL/Consultas/12.png}
     \end{figure}
    %%% 12
      \item \textbf{Agentes que adjuntaron una fotografía con extensión jpg} \\
   Consulta:\\ \textsc{SELECT curp, nombre FROM agente\_telefonico\\ WHERE agente\_telefonico.fotografia LIKE '\%jpg';}\\
     %%%%%%%%%%%% 14
    Existen 52 agentes que la adjuntaron  con la extensión \textit{jpg}, se muestra su CURP y  nombre.
     \begin{figure}[h]
         \centering
         \includegraphics[scale=.33]{PROYECTOFINAL/Consultas/14.png}
     \end{figure}
   \newpage 
    \item \textbf{Persona de contacto cuales nombres inician con la letra A o B} \\
Consulta:\\ \textsc{ SELECT * FROM cliente\\ WHERE (SELECT SUBSTRING (cliente.persona\_contacto, 1, 1) ILIKE 'A') OR (SELECT SUBSTRING (cliente.persona\_contacto, 1, 1) ILIKE 'B');} \\Existen 13 valores con las características requeridas.
     \begin{figure}[h]
         \centering
         \includegraphics[scale=.4]{PROYECTOFINAL/Consultas/13.png}
     \end{figure}
     %%%%%%%%% 13
    
    \item \textbf{Salas de entrenamiento en el edificio 1 o piso 5 de cualquier edificio}\\
     Consulta:\\ \textsc{ SELECT * FROM sala\_entrenamiento\\ WHERE sala\_entrenamiento.id\_edificio = 1\\ OR sala\_entrenamiento.id\_piso = 5;
}%%15
\begin{figure}[h]
    \centering
    \includegraphics[scale=.5]{PROYECTOFINAL/Consultas/15.png}
\end{figure}
     
    
\end{enumerate}
\end{document}